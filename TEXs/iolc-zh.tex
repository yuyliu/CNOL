% Packages and settings
\usepackage{xeCJK}
  \renewcommand\noindent{\hspace{-\parindent}}
  \vfuzz=1em
\usepackage{zhnumber}
\setCJKmainfont[
  Path=OTFs/Hani/,
  Extension=.otf,
  BoldFont=NotoSerifCJKsc-Bold,
  ItalicFont=AdobeKaitiStd-Regular,
  SlantedFont=AdobeFangsongStd-Regular
]{NotoSerifCJKsc-Regular}
\setCJKsansfont{Noto Sans CJK SC}
  \makeatletter
  \renewcommand\paragraph{%
    \@startsection{paragraph}{4}{\z@}%
                  {3.25ex \@plus1ex \@minus.2ex}%
                  {0pt}%
                  {\normalfont\normalsize\bfseries}}
  \makeatother
  \textwidth=40em
% Punctuation marks
\def \Pa #1{(#1)}
\def \Cl {:}
\def \Cm {,}
\def \Ec {、}
\def \Pr {。}
\def \Pt {.}
\def \Md {——}
\def \Nd {—}
\def \Sc {;}
\def \Sp {}
\def \Ts {}
\def \Ws { }
% Words and word combinations
\def \Iso {zh}
\def \Tsp #1{国际语言学奥林匹克中国区#1样题}
\def \Hsp #1{国际语言学奥林匹克中国区#1样题}
\def \Tct #1#2{第\zhnumber{#2}届国际语言学奥林匹克中国区#1}
\def \Hct #1#2{第\zhnumber{#2}届国际语言学奥林匹克中国区#1}
\def \Mdy #1#2#3{$\mbox{#3}$年$\mbox{#1}$月$\mbox{#2}$日}
\def \Opr {初选}
\def \Itr {终选}
\def \Ivc {个人赛}
\def \Tmc {团队赛}
\def \Pro {题目}
\def \Prs {题目}
\def \Sol {解答}
\def \Sls {解答}
\def \Ans {答题纸}
\def \Doc #1#2{#1#2}
\def \Pri #1{第\zhnumber{#1}题}
\def \Pts #1{$\mbox{#1}$分}
\def \Atc #1{#1}
\def \Atv #1{#1}
\def \Nos {数}
\def \Aeq {算术等式}
\def \Aes {算术等式}
\def \Eqs {等式}%(1—\arabic{index})}
\def \Wds {单词}
\def \Wcs {词组}
\def \Sfs {句子片段}
\def \Stc {句子}
\def \Sts {句子}
\def \Inl #1#2{#2#1}
\def \Ptl #1{部分#1}
\def \Wri #1#2{#2书写的#1}
\def \Wrw #1#2{#2写出的#1}
\def \Wro #1#2{#2写出的#1}
\def \Ins #1{用#1}
%\def \Hpy {汉语拼音}
\def \And #1#2{#1和#2}
\def \Bnd #1#2{#1并#2}
\def \Cnd #1#2{#1且#2}
\def \Que #1#2{#1及#2}
\def \Atq #1{和#1}
\def \Iao #1{乱序排列的#1}
\def \Sua #1{其#1}
\def \Ipa {国际音标转写}
\def \Trs #1{#1翻译}
\def \Nog #1{不大于$\mbox{#1}$}
\def \Lss #1{小于$\mbox{#1}$}
\def \Grt #1{大于$\mbox{#1}$}
\def \Nol #1{不小于$\mbox{#1}$}
\def \Ths #1{这些#1}
\def \One #1{#1中的一个}
\def \Etc {等地}
\def \Eds {编者}
\def \Txt {汉语文本}
\def \Eic {主编}
\def \Ted {技术编辑}
\def \Nam {姓名}
\def \Stn {编号}
\def \Prn {题号}
\def \Pgn {页码}
\def \Dne {不存在}
\def \Otw {其他情况}
\def \Aft #1{在#1后}
\def \Nas {鼻音}
\def \Nsl {鼻音化}
\def \Stm {词根}
\def \Plr {复数}
\def \Acc {宾格}
\def \Cmp {比较格}
\def \Loc {方位格}
\def \Nom {主格}
% Sentences
\def \Rule {除非题目明确说明你无需这样做,你应描述你在语料中发现的任何规律或规则。}
\def \Gbis #1{以下是一个#1:}
\def \Gbre #1{以下是一些#1:}
\def \Giao #1{以下是乱序排列的一些#1:}
\def \Dtma {将之正确配对。}
\def \Dtmb {正确配对:}
\def \Fila {填写其中的空缺。}
\def \Filb {填写空缺:}
\def \Trla #1#2{将#1翻译成#2。}
\def \Trlb #1{翻译成#1:}
\def \Wrna #1{用阿拉伯数字写出#1。}
\def \Wrnb {用阿拉伯数字写出:}
\def \Wrwa #1#2{用#2写出#1。}
\def \Wrwb #1{用#1写出:}
\def \Wroa #1#2{用#2写出#1。}
\def \Wrob #1{用#1写出:}
\def \Expa {解释你的答案。}
\def \Alle #1{在这些等式中出现的所有数都#1。}
\def \Allp #1{在该题中出现的所有数都#1。}
\def \Coin #1#2{#1的答案应与#2一致。}
\def \Noex {除答案外,任何解释皆不必要,亦不予评分。}
\def \Isli #1{#1是孤立语言。}
\def \Infm #1#2{#1属于#2语系。}
\def \Inbr #1#2#3{#1属于#3语系#2语族。}
\def \Ingr #1#2#3{#1属于#3语系#2语支。}
\def \Hist #1#2{约$\mbox{#2}$年前,#1使用该语言。}
\def \Glob #1{在世界各地,约有$\mbox{#1}$人使用该语言。}
\def \Stat #1#2{在#2,约有$\mbox{#1}$人使用该语言。}
\def \Ethn #1#2#3{在#3,约有$\mbox{#1}$名#2族人使用该语言。}
\def \Isac #1{#1是辅音。}
\def \Arec #1{#1是辅音。}
\def \Isav #1{#1是元音。}
\def \Arev #1{#1是元音。}
\def \Indi #1#2{标记$\text{#1}$表示#2。}
\def \Long {标记$\text{\D{ː}}$表示前面的元音是长音。}
\def \Dvdl {元音双写表示长音。}
\def \Gdlk {祝你好运!}
% Languages and scripts
\def \TXT {\ZHO}
\def \ENG {英语}
\def \ZHO {汉语}
\def \LATN {拉丁字母}
% Countries
\def \CN {中国}
% People
\def \HhLi {\mbox{李惠涵}}
\def \YeLq {\mbox{刘冶}}
\def \YyLq {\mbox{刘羽扬}}
\def \YmLo {\mbox{罗一茗}}
\def \YpSp {\mbox{孙毓培}}
\def \ZjWh {\mbox{王子佶}}
\def \JnXu {\mbox{徐进}}
\def \LnYe {\mbox{叶琳}}
\def \SyYu {\mbox{俞舒悦}}
\def \LhVg {\mbox{郑灵辉}}
